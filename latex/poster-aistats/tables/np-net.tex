
\def\layersep{8cm}
\def\nodesep{3.3cm}
\def\halfshift{1.65cm}
\def\nodesize{2.5cm}

\newcommand{\numdims}[0]{3}
\newcommand{\numhidden}[0]{4}
\newcommand{\upnodedist}[0]{2.3cm}

\newcommand{\neuronfunc}[2]{
\FPeval{\result}{clip(#1+#2)}
%
\null\hspace*{-0.5cm}\includegraphics[width=1.9cm, clip, trim=10mm 0mm 10mm 0mm]{../../figures/two-d-draws/sqexp-draw-\result}
%\end{pgfonlayer}
}


\begin{tabular}{c}
\begin{tikzpicture}[draw=black, node distance=\layersep]
    \tikzstyle{interarrows}=[->, ultra thick, -biggertip, black!50, line width = 2pt, shorten >= -0cm, shorten <= -0cm]
    \tikzstyle{neuron}=[circle, draw = black, inner sep=-0pt, line width = 2pt]
    \tikzstyle{input neuron}=[neuron, minimum size=\nodesize];
    \tikzstyle{output neuron}=[neuron, minimum size=\nodesize];
    \tikzstyle{hidden neuron}=[neuron, minimum size=\nodesize];
    \tikzstyle{annot} = [text width=4em, text centered]

    % Draw the input layer nodes
    \foreach \name / \y in {1,...,\numdims}
        \node[input neuron] (I-\name) at (0,-\nodesep*\y) {$x_\y$};

    % Draw the hidden layer nodes
    \foreach \name / \y in {1,...,\numhidden}
        \path[yshift=\halfshift] node[hidden neuron] (H-\name) at (\layersep,-\nodesep*\y) { \neuronfunc{\y}{0}};

    % Draw the hidden layer nodes
    \foreach \name / \y in {1,...,\numhidden}
        \path[yshift=\halfshift] node[hidden neuron] (H2-\name) at (2*\layersep,-\nodesep*\y) {\neuronfunc{\y}{4}};

    % Draw the output layer node
    \foreach \name / \y in {1,...,\numdims}
    	\node[output neuron] (O-\name) at (3*\layersep,-\nodesep*\y) {\neuronfunc{\y}{8}};

    % Connect every node in the input layer with every node in the hidden layer.
    \foreach \source in {1,...,\numdims}
        \foreach \dest in {1,...,\numhidden}
            \path (I-\source) edge[interarrows] (H-\dest);
            
    \foreach \source in {1,...,\numhidden}
        \foreach \dest in {1,...,\numhidden}
            \path (H-\source) edge[interarrows] (H2-\dest);            

    % Connect every node in the hidden layer with the output layer
    \foreach \source in {1,...,\numhidden}
        \foreach \dest in {1,...,\numdims}
    	    \path (H2-\source) edge[interarrows] (O-\dest);

    % Annotate the layers
    \node[annot,above of=I-1, node distance=\upnodedist] {Inputs};
    \node[annot,below of=I-\numdims, node distance=\upnodedist] {$\vx$};    
    \node[annot,above of=H-1, node distance=\upnodedist, text width = 7cm] {Hidden Layer};
    \node[annot,above of=H2-1, node distance=\upnodedist, text width = 7cm] {Hidden Layer};
    \node[annot,below of=H-\numhidden, node distance=2.5cm, text width = 7cm] {$\vf^{(1)}(\vx)$};
    \node[annot,below of=H2-\numhidden, node distance=2.5cm, text width = 7cm] {$\vf^{(2)}(\vf^{(1)}(\vx))$};
    \node[annot,above of=O-1, node distance=\upnodedist] {Outputs};
    \node[annot,below of=O-\numdims, node distance=2.5cm, text width = 7cm] {$\vy$};
\end{tikzpicture}
\end{tabular}


