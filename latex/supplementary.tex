\documentclass{article}
\usepackage{include/nips13submit_e,times}
%\input{include/commenting.tex}
\usepackage{include/preamble}


\title{On the Inductive Bias of Deep Neural Networks}


\author{
Pittsburgh, PA 15213 \\
\texttt{hippo@cs.cranberry-lemon.edu} \\
\And
Coauthor \\
Affiliation \\
Address \\
\texttt{email} \\
\AND
Coauthor \\
Affiliation \\
Address \\
\texttt{email} \\
}

% Custom notation
\newcommand{\fdeep}{f^{1:L}}
\newcommand{\flast}{f^{L}}
\newcommand{\Jx}{J_{\vx \rightarrow \vy}}
\newcommand{\Jxx}{J_{\vx \rightarrow \vy}(\vx)}
\newcommand{\Jy}{J_{\vy \rightarrow \vx}}
\newcommand{\Jyy}{J_{\vy \rightarrow \vx}(\vy)}
\newcommand{\detJyy}{ \left| J_{\vy \rightarrow \vx}(\vy) \right|}


% The \author macro works with any number of authors. There are two commands
% used to separate the names and addresses of multiple authors: \And and \AND.
%
% Using \And between authors leaves it to \LaTeX{} to determine where to break
% the lines. Using \AND forces a linebreak at that point. So, if \LaTeX{}
% puts 3 of 4 authors names on the first line, and the last on the second
% line, try using \AND instead of \And before the third author name.

\newcommand{\fix}{\marginpar{FIX}}
\newcommand{\new}{\marginpar{NEW}}

%\nipsfinalcopy % Uncomment for camera-ready version

\begin{document}


\maketitle

\begin{abstract}
We examine properties of deep Gaussian processes, a nonparametric prior over infinitely-wide neural networks with arbitrarily many hidden layers.  We argue that these results shed light on the types of functions which will be easy to learn by more conventional deep neural networks.
\end{abstract}

\section{Introduction}

Deep networks have become an important tool for machine learning [cite].  However, training these models are difficult, Many arguments have been made for the need for deep architectures [cite Bengio].  However, it is hard to know what effect the deepness of an architecture has.  Also, the weights don't necessarily move that much from their initialization.

We introduce a generative non-parametric model to address this problem.  Our approach is based on the GP-LVM ~\cite{lawrence2004gaussian,salzmann2008local,lawrence2009non}, a flexible nonparametric density model.

\paragraph{Approach}
We are going to analyze these networks by looking at the neighbourhood surrounding random points.

\section{Deep Gaussian Processes}

\subsection{Gaussian Processes}

\subsection{Gaussian Process Latent Variable Model}

\begin{figure}
\centering
\includegraphics[width=0.8\columnwidth]{figures/gplvm_1d_draw_8} 
\caption{A draw from a Gaussian process latent variable model.  Bottom:  The latent datapoints $\vX$ are distributed according to a parametric base distribution (a Gaussian).  Top right:  A smooth function $f$ drawn from a Gaussian process prior is applied to obtain $\vY$ = $f(\vX)$.  Left:  The observed data $\vY$ is distributed according to a non-Gaussian density.}
\label{fig:gplvm_intro}
\end{figure}

The GP-LVM specifies a model wherein latent variables $\vX$ are warped by an unknown smooth, function $f$ to produce the observed data $\vY$.  The prior used over functions in the GP-LVM is the Gaussian process~\cite{rasmussen38gaussian}.

While not typically thought of as a density model, the GPLVM does define a nonparametric density over observations~\cite{nickisch2010gaussian}.   Figure \ref{fig:gplvm_intro} demonstrates how a Gaussian latent density, when warped by a random smooth function, can give rise to a non-Gaussian density in the observed space.

The dimension of the observed data ($D$) doesn't need to match the dimension of the latent space ($Q$).  When $Q$ is 2 or 3, the GP-LVM can also be used for visualization of high-dimensional data.  The mapping from $\vX$ to each dimension of the observed data is assumed to be independent, so the likelihood has a simple form which implicitly integrates over $f$:
%
\begin{align}
p(\vY | \vX,\bm{\theta})  = (2 \pi)^{-\frac{DN}{2}}  |\vK|^{-\frac{D}{2}} \exp \left( -\frac{1}{2} {\rm tr}( \vY^{\top} \vK^{-1} \vY) \right),
\label{eq:py_x}
\end{align}
where $\vK$ is the $N \times N$ covariance matrix defined 
by the kernel function $k(\vx_{n},\vx_{m})$,
and $\bm{\theta}$ is the kernel hyperparameter vector.
In this paper, we use an RBF kernel with an additive noise term:
\begin{align}
k(\vx_{n},\vx_{m}) &= \alpha \exp\left( - \frac{1}{2 \ell^2}(\vx_n - \vx_m)^{\top} (\vx_n - \vx_m) \right) + \delta_{nm} \beta^{-1}.
\end{align}

\subsection{Different settings}

In our analysis, we assume that functions are normalized.



\begin{itemize}
\item The kernel is an RBF, meaning that all functions are infinitely differentiable.  This also corresponds to the infinite limit taken by Radford Neal [cite]
\item The dimension of the latent space is always the same.
\end{itemize}

We consider several different variations of deep GPs:

\begin{itemize}
\item mean function $m_d(x_d) = 0$
\item mean function $m_d(x_d) = x_d$
\end{itemize}

\paragraph{Noise versus no noise}
Do the deep functions we're interested in modeling have noise added at each step?  Consider the example of handwritten digits.  One putative model would be that $x$ is the number that the human is trying to write, then a series of functions of his nervous system and arm (with feedback loops) cascade to produce the observed digit.  This whole process can be expected to have small amounts of noise at each step, but presumably \emph{can only work if the amount of noise is small}.

A function whose result gets a lot of noise added at each intermediate step is probably not useful unless there is an error-correcting step applied downstream.  Perhaps our results hold because random functions do not do error-correction.  Perhaps this is another intuition for why 'dropout' works - functions are explicitly trained to be robust to noise.  Denoising autoencoders are also trained in this way.

\begin{itemize}
\item Density models
\item Function models
\end{itemize}

\section{Related Work}

[Ryan Adams, Wallach and Zoubin on nonparametric deep nets]

[Deep GP Kernels by Youngmin Cho]
\url{http://cseweb.ucsd.edu/~yoc002/paper/thesis_youngmincho.pdf}

[GP Dynamical systems]

[Warping a 1d uniform distribution]

Layer-wise analysis of deep networks with Gaussian
kernels:
\url{http://books.nips.cc/papers/files/nips23/NIPS2010_0206.pdf}

\section{Main Results}

\paragraph{Stationarity} When characterizing deep \gp{}s, having a stationary kernel means that expectations of our function will be the same no matter which point we evaluate it at.  In other words, for any statistic $S(f(\vx))$, $\expect [ S( f(\vx) ] = \expect [ S( f(\vx') ] \forall \vx \forall \vx'$.  
%This is not true in general for deep \gplvm{}s.




%\begin{proposition}
\paragraph{The mean of a deep zero-mean \gp{} has mean zero.}
%The mean of a deep zero-mean \gp{} has mean zero.
%\end{proposition}
%
%\begin{proof}[Proof]
All functions are drawn \iid, so we can ignore all but the last transformation $f_L$.  Since $f_L$ is zero-mean, then $\forall \vx \forall d, \expectargs{\GP}{f_d(x)} = 0 $.
%\end{proof}


\subsection{Properties of the Jacobian}

\paragraph{The derivatives of a function drawn from a \gp{} prior with an isotropic SE kernel are \iid Normal}

Because differentiation is a linear operator, the derivatives of a function drawn from a \gp{} prior are also jointly Gaussian distributed, with covariance between derivatives w.r.t. different dimensions of $\vx$ given by:
%
\begin{align}
\cov \left( \frac{\partial f(\vx)}{\partial x_{d_1}}, \frac{\partial f(\vx)}{\partial x_{d_2}} \right) =
\frac{\partial^2 k(\vx, \vx'))}{\partial x_{d_1} \partial x_{d_2}'} \bigg|_{\vx=\vx'}
\end{align}
%
[cite carl's paper?] 
%
If our kernel is a product over individual dimensions $k(\vx, \vx') = \prod_d^D k_d(x_d, x_d')$, as in the case of the isotropic squared-exp kernel, then the diagonal covariances are given by $\frac{\sigma_o^2}{\ell^2}$, and the off-diagonal entries are zero.  This means that elements are independent and identically distributed.

\paragraph{The elements of the Jacobian of a \gp{} with an isotropic SE kernel are \iid Gaussians}

The Jacobian of the $\ell$\asdf function is:
%
\begin{align}
\Jx^\ell(\vx) =\begin{bmatrix} \dfrac{\partial f^\ell_1 (\vx) }{\partial x_1} & \cdots & \dfrac{\partial f^\ell_1 (\vx)}{\partial x_D} \\ \vdots & \ddots & \vdots \\ \dfrac{\partial f^\ell_D (\vx)}{\partial x_1} & \cdots & \dfrac{\partial f^\ell_D (\vx)}{\partial x_D}  \end{bmatrix}
\end{align}
%
Because we've assumed that the \gp{} on each output dimension $f_d(\vx) \sim \GP$ is independent, it follows that for a given $\vx$, each row of $\Jxx$ is independent.
Above, we showed that the elements of each row are independent.
This means that each entry in the Jacobian of a \gp{}-distributed transformation is \iid Normal.

We also have that if $\vx = f(\vy)$, then $\Jy\inv(\vy) = \Jxx$.

\paragraph{The Jacobian of a deep \gp{} is a product of random normal matrices}
By the multivariate chain rule, the derivative (Jacobian) of any compositions of functions is simply the product of the Jacobians of each function.  
%
and the Jacobian of the composed (deep) function is:
%
\begin{align}
J^{1:L}(x) 
% = \frac{\partial \fdeep(x) }{\partial x} 
%= \frac{\partial f^1(x) }{\partial x} \frac{\partial f^2(x) }{\partial f^1(x)} \cdots \frac{\partial f^L(x) }{\partial f^{L-1}(x)}
= \prod_{\ell = 1}^{L} J^L (x)
\end{align}

\paragraph{The density in the observed space of a deep density}

Let $\vy = \fdeep(\vx)$ be a random variable.
%
The change-of-variables formula is:
\begin{align}
p(f(\vx)) = \sum_{k=1}^{n(\vy_k = f(\vx_k))} \left| \frac{d}{dy} f^{-1}(\vy_{k}) \right| \cdot p_x(\vx_{k})
\end{align}
%
Assuming that $\fdeep(\vx)$ is one-to-one,
\begin{align}
p(f(\vx)) = p_x( \vx ) \left| \frac{  \partial f\inv(x) }{\partial x } \right|
\end{align}
%
We can use that
\begin{align}
\det(A\inv) = \frac{1}{\det(A)}
\end{align}
%
Assuming that $\fdeep(\vx)$ is one-to-one,
\begin{align}
p(f(\vx)) = p_x( \vx ) \left| J\inv(\vx) \right| = p_x( \vx ) \frac{1}{\left| J(\vx) \right|}
\end{align}

\paragraph{Eigenvalues of inverse}  We can try to prove things about the eigenvalues of $J(\vx)$.  However, in order to characterize the density of $p(\vy)$, we will need to analyse the eigenspectrum of $J\inv(\vx)$.  Helpfully, if $\lambda$ are the eigenvalues of $J$, then $\lambda \inv$ are the eigenvalues of $J\inv$.

\paragraph{Determinant in terms of Eigenvalues}  If $\lambda$ are the eigenvalues of J, then
\begin{align}
|J| = \prod_i \lambda_i
\end{align}

\paragraph{The Hessian of a deep density model}

Since we know the density of a point drawn from a deep GP, we can also look at the local curvature through the Hessian.  Given an $\vx, \vy$ pair $\vy = f(\vx)$, 
%
\begin{align}
p(\vy) 
= \frac{p_x( f\inv(\vy ))}{\left| J( f\inv(\vy)) \right|} 
= \frac{p_x( \vx )}{\left| J( \vx) \right|}
= p_x( \vx ) \left| J( \vx)\inv \right|
\end{align}
%
So we can say that the determinant of the inverse transform $\left| J( \vx)\inv \right|$ defines the local distortion of density.

We want to know how many directions we can move in.

We could characterize this by the probability, if we moved in a random direction, of not moving into a region of low probability.  If we're at a local maximum, we can ask how many eigenvalues of the Hessian are small.


The Hessian of the determinant of the Jacobian is:
\begin{align}
H_y \left( \left| J_{\vy \rightarrow \vx} (\vy) \right| \right)
 & = \begin{bmatrix}
\dfrac{\partial^2 \detJyy}{\partial y_1^2} & \dfrac{\partial^2 \detJyy}{\partial y_1\,\partial y_2} & \cdots & \dfrac{\partial^2 \Jyy}{\partial y_1\,\partial y_n} \\[2.2ex]
\dfrac{\partial^2 \detJyy}{\partial y_2\,\partial y_1} & \dfrac{\partial^2 \detJyy}{\partial y_2^2} & \cdots & \dfrac{\partial^2 \detJyy}{\partial y_2\,\partial y_n} \\[2.2ex]
\vdots & \vdots & \ddots & \vdots \\[2.2ex]
\dfrac{\partial^2 \detJyy}{\partial y_n\,\partial y_1} & \dfrac{\partial^2 \detJyy}{\partial y_n\,\partial y_2} & \cdots & \dfrac{\partial^2 \detJyy}{\partial y_n^2}
\end{bmatrix} \\
 & = H_y \left( \left| \Jyy \right| \right) \\
 & = H_y \left( \prod_i \lambda^{\Jy}_i ( \vy) \right) \qquad \textrm{where $\lambda$ are eigenvalues of $\Jyy$}
 \\
 & = H_y \left( \prod_i \frac{1}{\lambda^{\Jx}_i ( \vx )} \right)
\end{align}
%
where $H_y$ means that the second derivatives in the Hessian are taken w.r.t. $\vy$, and $\lambda^{J\inv}_i ( \vx)$ are the eigenvalues of the Jacobian (the total derivative of $f\inv(\vy)$ w.r.t. $\vx$).

\paragraph{Derivative of determinant}
\begin{align}
\frac{\partial \det(A)}{\partial A_{ij}}= \operatorname{adj}(A)_{ji}= \det(A)(A^{-1}_{ji}) \\
\frac{\mathrm{d} \det(A)}{\mathrm{d} \alpha} =  \det(A) \operatorname{tr}\left(A^{-1} \frac{\mathrm{d} A}{\mathrm{d} \alpha}\right)
\end{align}
%
In our case, we have:
%
\begin{align}
\frac{\partial \det(\Jyy)}{\partial \vy_i} & = \det(\Jy) \trace \left(\Jx \frac{\partial \Jyy}{\partial \vy_i}\right) \\
& = \det(\Jy) \trace \left( - \Jx \Jy \frac{\partial \Jx(\vy)}{\partial \vy_i} \Jy \right) \\
& = \det(\Jy) \trace \left( - \frac{\partial \Jx(\vy)}{\partial \vy_i} \Jy \right) \\
& = \det(\Jy) \trace \left( - \frac{\partial \Jx(\vx)}{\partial \vx} \frac{\partial \vx}{\partial \vy_i} \Jy \right) \\
& = \det(\Jy) \trace \left( - \frac{\partial \Jx(\vx)}{\partial \vx} \Jy^{:,i} \Jy \right) \\
& = \det(\Jy) \trace \left( - \Jy \frac{\partial \Jx(\vx)}{\partial \vx} \Jy^{:,i} \right) \label{eqn:ddet_tractable} \\
& = \det(\Jy) \trace \left( \frac{\partial \Jy(\vy)}{\partial \vx} \right)
\label{eqn:ddet_simple}
\end{align}
%
We don't have an analytic form for \eqref{eqn:ddet_simple}, but we can compute \eqref{eqn:ddet_tractable} if we can compute the term $\frac{\partial \Jx(\vx)}{\partial \vx}$, which is just the second derivative of $\fdeep(\vx)$.

\paragraph{Derivative of SVD}
a

\url{http://www.ics.forth.gr/_publications/2000_eccv_SVD_jacobian.pdf} says that:
%
\begin{align}
\frac{\partial \det(A)}{\partial A_{ij}}= \operatorname{adj}(A)_{ji}= \det(A)(A^{-1}_{ji})
\end{align}



\paragraph{Hessian of determinant}
\begin{align}
\frac{\partial^2 \det(A)}{\partial A_{ij}\partial A_{mn}}
& = \det(A)(A^{-1}_{ji}A^{-1}_{nm} - A^{-1}_{ni}A^{-1}_{jm})
\end{align}

We can also say that

\url{http://math.stackexchange.com/questions/50386/the-hessian-of-the-determinant}

\begin{align}
\frac{\partial^2 \det(A(\vy))}{\partial y_i \partial y_j}
& = \frac{\partial^2 \det(A(\vy))}{\partial A_{ij}\partial A_{mn}} \left( \frac{\partial A_{ij}}{\partial y_i} + \frac{\partial A_{ij}}{\partial y_j} \right) \left( \frac{\partial A_{mn}}{\partial y_i} + \frac{\partial A_{mn}}{\partial y_j} \right)
%& = \det(A)(A^{-1}_{ji}A^{-1}_{nm} - A^{-1}_{ni}A^{-1}_{jm})
\end{align}

or we can try:

\begin{align}
\frac{\partial^2 \det(A(\vy))}{\partial y_i \partial y_j}
& = \frac{\partial }{\partial \vy_j} \det(\Jy) \trace \left( \frac{\partial \Jy(\vy)}{\partial \vx} \right) \\
& = \det(\Jy) \frac{\partial }{\partial \vy_j} \trace \left( \frac{\partial \Jy(\vy)}{\partial \vx} \right) 
  + \trace \left( \frac{\partial \Jy(\vy)}{\partial \vx} \right) \frac{\partial }{\partial \vy_j} \det(\Jy) \\
\end{align}

or:

\begin{align}
\frac{\partial^2 \det(A(\vy))}{\partial y_i \partial y_j}
= \det(A) & \Big[ 
          \trace \left( A\inv \frac{\partial^2 A(\vy)}{\partial \vy_i \partial \vy_j } \right) \nonumber \\
          & + \trace \left( A\inv \frac{\partial A(\vy)}{\partial \vy_i} \right) 
            \trace \left( A\inv \frac{\partial A(\vy)}{\partial \vy_j} \right) \nonumber \\
           & + \trace \left( A\inv \frac{\partial A(\vy)}{\partial \vy_i} 
                          A\inv \frac{\partial A(\vy)}{\partial \vy_j} \right)
         \Big] \\
\frac{\partial^2 \det(\Jy(\vy))}{\partial y_i \partial y_j}
= \det(\Jy) & \Big[
          \trace \left( \Jy\inv \frac{\partial^2 \Jy(\vy)}{\partial \vy_i \partial \vy_j } \right) \nonumber \\
          & + \trace \left( \Jy\inv \frac{\partial \Jy(\vy)}{\partial \vy_i} \right) 
            \trace \left( \Jy\inv \frac{\partial \Jy(\vy)}{\partial \vy_j} \right) \nonumber \\
           & + \trace \left( \Jy\inv \frac{\partial \Jy(\vy)}{\partial \vy_i} 
                          \Jy\inv \frac{\partial \Jy(\vy)}{\partial \vy_j} \right)
         \Big] \\
= \det(\Jy) & \Big[
          \trace \left( \Jx \frac{\partial^2 \Jy(\vy)}{\partial \vy_i \partial \vy_j } \right) \nonumber \\
          & + \trace \left( \Jx \frac{\partial \Jy(\vy)}{\partial \vy_i} \right) 
            \trace \left( \Jx \frac{\partial \Jy(\vy)}{\partial \vy_j} \right) \nonumber \\
           & + \trace \left( \Jx \frac{\partial \Jy(\vy)}{\partial \vy_i} 
                          \Jx \frac{\partial \Jy(\vy)}{\partial \vy_j} \right)
         \Big] \\        
%
%\frac{\partial^2 \det(A(\vy))}{\partial A_{ij}\partial A_{mn}} \left( \frac{\partial A_{ij}}{\partial y_i} + \frac{\partial A_{ij}}{\partial y_j} \right) \left( \frac{\partial A_{mn}}{\partial y_i} + \frac{\partial A_{mn}}{\partial y_j} \right)
%& = \det(A)(A^{-1}_{ji}A^{-1}_{nm} - A^{-1}_{ni}A^{-1}_{jm})
\end{align}

\paragraph{Eigenvalues of Hessian}


\subsection{Entropy}

Differential entropy:
%
\begin{align}
h(X) = -\int_\mathbb{X} p(x)\log p(x)\,dx
\end{align}
%
From \url{http://www.mtm.ufsc.br/~taneja/book/node13.html}:
``The entropy of a discrete random variable remains invariant under a change of variable, however with a continuous random variable the entropy does not necessarily remain invariant.''

\paragraph{The expected entropy of $p(f(\vx))$ increases with the depth of $f$}


Maybe use mutual information to prove this?


\paragraph{Analogy to a Gaussian warped by successive random linear transformations.}

if $\vx \sim \Nt{\vmu}{\vSigma}$, then a linear transformation of $\vy = A \vx \sim \Nt{A \vmu}{ A \vSigma A\tra}$


\section{Random Matrix Theory}

``On the number of real eigenvalues of products of random matrices and an application to quantum entanglement''
\url{http://arxiv.org/pdf/1301.7601v1.pdf}


\subsection{Filamentation}

\paragraph{Attempted definition of a filament}  A continuous, twice-differentiable region of a pdf is called a \emph{filament} to the degree that, weighted by density, only one eigenvalue of the Hessian of the pdf are is small relative to the average eigenvalue.

\begin{figure}
\centering
\begin{tabular}{ccc}
\includegraphics[width=0.3\columnwidth]{figures/deep_draws/deep_gp_sample_layer_1} &
\includegraphics[width=0.3\columnwidth]{figures/deep_draws/deep_gp_sample_layer_2} &
\includegraphics[width=0.3\columnwidth]{figures/deep_draws/deep_gp_sample_layer_3} \\
$p(\vx)$ & $p(f_1(\vx))$ & $p(f_2(f_1(\vx)))$ \\ \\
\includegraphics[width=0.3\columnwidth]{figures/deep_draws/deep_gp_sample_layer_4} &
\includegraphics[width=0.3\columnwidth]{figures/deep_draws/deep_gp_sample_layer_5} &
\includegraphics[width=0.3\columnwidth]{figures/deep_draws/deep_gp_sample_layer_6} \\
$p(f_3(f_2(f_1(\vx))))$ & $p(f_4(f_3(f_2(f_1(\vx)))))$ & $p(f_5(f_4(f_3(f_2(f_1(\vx)))))$
\end{tabular}
\caption{Draws from a deep GP.  A distribution is warped by successive functions drawn from a \gp{} prior.}
\label{fig:filamentation}
\end{figure}


\begin{align}
\textrm{Filament}(p(\vx)) = 1 - \int \frac{ \bar \lambda}{ \lambda_\textrm{min}} p(\vx) d\vx
\end{align}

where $\lambda_\textrm{min}$ is the smallest eigenvalue of the Hessian of $p(\vx)$.

There is some sort of connection between average eigenvalue and signal power:
[\url{http://www1.i2r.a-star.edu.sg/~yhzeng/ZENG-TCOM-09-0402.pdf}]

This definition doesn't depend on scale, although it does depend on parameterization in general.

\paragraph{Example: A uniform distribution of any dimension}
A uniform distribution has $\textrm{Filament}(p(\vx)) = 0$.

\paragraph{Example: Any distribution defined over 1 dimension}
In one dimension, there is only one eigenvalue, so $ \bar \lambda = \lambda_\textrm{min}$, and thus $\frac{ \bar \lambda}{ \lambda_\textrm{min}} = 1$ everywhere.  So every 1D distribution has $\textrm{Filament}(p(\vx)) = 0$.

\paragraph{Example: A 1D sigmoid in two dimensions}
A 1d sigmoidal-shaped distribution in two dimensions will have $\textrm{Filament}(p(\vx)) > 0$, because at the top of the curve, $\frac{ \bar \lambda}{ \lambda_\textrm{min}} > 1$, and this can't be cancelled out, because everywhere else, $\frac{ \bar \lambda}{ \lambda_\textrm{min}} \geq 1$.

\subsection{Why does filamentation occur?}
Intuition: The space will be squished in some directions, and stretched in others.  However, any direction's Jacobian will be a product of lots of directional Jacobians.  This distribution will become heavy-tailed, meaning that it will be dominated by a few small values.  We can characterize the ratio of the maximum direction of curvature to the average.




\section{Relation to Deep Neural Networks}

There are two reasons to think that the deep-GP prior is related to the inductive bias of deep neural networks.  First, the 

\paragraph{Weights don't change much during training}
James Martens says that the weights don't change much during training.  Perhaps we could make a plot showing the original weights versus the trained weights?

\paragraph{$L_2$ regularization of weights corresponds in a loose sense to independent Gaussian priors on the weights}
Which correspond to Gaussian processes (Neal)

\section{To what extent is filamentization a pathology?}

Points close in $x$-space can be very far in $y$-space, and vice versa.

\section{Conclusions we might like to be able to make}

\paragraph{Deep neural networks and deep Gaussian processes are analyzable using random matrix theory.}  After proving that the Jacobian is an i.i.d. Gaussian matrix, many other forms of 

\paragraph{If you want to use very deep nets, you won't be able to do so if you initialize/regularize all your weights independently}  We might want to think about different ways of 

\paragraph{If you initialize independently, the density becomes fractal} Points close in $x$-space can be very far in $y$-space, and vice versa.

\paragraph{A spikey eigenspecturm will lead to saturation}
Maybe we should initialize differently in order to avoid such saturation, like Martens' sparse initialization: \url{http://www.cs.toronto.edu/~jmartens/docs/Deep_HessianFree.pdf}

Or, maybe we should train differently to 








\paragraph{The Hessian of a deep density model}

Since we know the density of a point drawn from a deep GP, we can also look at the local curvature through the Hessian.  Given an $\vx, \vy$ pair $\vy = f(\vx)$, 
%
\begin{align}
p(\vy) 
= \frac{p_x( f\inv(\vy ))}{\left| J( f\inv(\vy)) \right|} 
= \frac{p_x( \vx )}{\left| J( \vx) \right|}
= p_x( \vx ) \left| J( \vx)\inv \right|
\end{align}
%
So we can say that the determinant of the inverse transform $\left| J( \vx)\inv \right|$ defines the local distortion of density.

We want to know how many directions we can move in.

We could characterize this by the probability, if we moved in a random direction, of not moving into a region of low probability.  If we're at a local maximum, we can ask how many eigenvalues of the Hessian are small.


The Hessian of the determinant of the Jacobian is:
\begin{align}
H_y \left( \left| J_{\vy \rightarrow \vx} (\vy) \right| \right)
 & = \begin{bmatrix}
\dfrac{\partial^2 \detJyy}{\partial y_1^2} & \dfrac{\partial^2 \detJyy}{\partial y_1\,\partial y_2} & \cdots & \dfrac{\partial^2 \Jyy}{\partial y_1\,\partial y_n} \\[2.2ex]
\dfrac{\partial^2 \detJyy}{\partial y_2\,\partial y_1} & \dfrac{\partial^2 \detJyy}{\partial y_2^2} & \cdots & \dfrac{\partial^2 \detJyy}{\partial y_2\,\partial y_n} \\[2.2ex]
\vdots & \vdots & \ddots & \vdots \\[2.2ex]
\dfrac{\partial^2 \detJyy}{\partial y_n\,\partial y_1} & \dfrac{\partial^2 \detJyy}{\partial y_n\,\partial y_2} & \cdots & \dfrac{\partial^2 \detJyy}{\partial y_n^2}
\end{bmatrix} \\
 & = H_y \left( \left| \Jyy \right| \right) \\
 & = H_y \left( \prod_i \lambda^{\Jy}_i ( \vy) \right) \qquad \textrm{where $\lambda$ are eigenvalues of $\Jyy$}
 \\
 & = H_y \left( \prod_i \frac{1}{\lambda^{\Jx}_i ( \vx )} \right)
\end{align}
%
where $H_y$ means that the second derivatives in the Hessian are taken w.r.t. $\vy$, and $\lambda^{J\inv}_i ( \vx)$ are the eigenvalues of the Jacobian (the total derivative of $f\inv(\vy)$ w.r.t. $\vx$).

\paragraph{Derivative of determinant}
\begin{align}
\frac{\partial \det(A)}{\partial A_{ij}}= \operatorname{adj}(A)_{ji}= \det(A)(A^{-1}_{ji}) \\
\frac{\mathrm{d} \det(A)}{\mathrm{d} \alpha} =  \det(A) \operatorname{tr}\left(A^{-1} \frac{\mathrm{d} A}{\mathrm{d} \alpha}\right)
\end{align}
%
In our case, we have:
%
\begin{align}
\frac{\partial \det(\Jyy)}{\partial \vy_i} & = \det(\Jy) \trace \left(\Jx \frac{\partial \Jyy}{\partial \vy_i}\right) \\
& = \det(\Jy) \trace \left( - \Jx \Jy \frac{\partial \Jx(\vy)}{\partial \vy_i} \Jy \right) \\
& = \det(\Jy) \trace \left( - \frac{\partial \Jx(\vy)}{\partial \vy_i} \Jy \right) \\
& = \det(\Jy) \trace \left( - \frac{\partial \Jx(\vx)}{\partial \vx} \frac{\partial \vx}{\partial \vy_i} \Jy \right) \\
& = \det(\Jy) \trace \left( - \frac{\partial \Jx(\vx)}{\partial \vx} \Jy^{:,i} \Jy \right) \\
& = \det(\Jy) \trace \left( - \Jy \frac{\partial \Jx(\vx)}{\partial \vx} \Jy^{:,i} \right) \label{eqn:ddet_tractable} \\
& = \det(\Jy) \trace \left( \frac{\partial \Jy(\vy)}{\partial \vx} \right)
\label{eqn:ddet_simple}
\end{align}
%
We don't have an analytic form for \eqref{eqn:ddet_simple}, but we can compute \eqref{eqn:ddet_tractable} if we can compute the term $\frac{\partial \Jx(\vx)}{\partial \vx}$, which is just the second derivative of $\fdeep(\vx)$.

\paragraph{Derivative of SVD}
a

\url{http://www.ics.forth.gr/_publications/2000_eccv_SVD_jacobian.pdf} says that:
%
\begin{align}
\frac{\partial \det(A)}{\partial A_{ij}}= \operatorname{adj}(A)_{ji}= \det(A)(A^{-1}_{ji})
\end{align}



\paragraph{Hessian of determinant}
\begin{align}
\frac{\partial^2 \det(A)}{\partial A_{ij}\partial A_{mn}}
& = \det(A)(A^{-1}_{ji}A^{-1}_{nm} - A^{-1}_{ni}A^{-1}_{jm})
\end{align}

We can also say that

\url{http://math.stackexchange.com/questions/50386/the-hessian-of-the-determinant}

\begin{align}
\frac{\partial^2 \det(A(\vy))}{\partial y_i \partial y_j}
& = \frac{\partial^2 \det(A(\vy))}{\partial A_{ij}\partial A_{mn}} \left( \frac{\partial A_{ij}}{\partial y_i} + \frac{\partial A_{ij}}{\partial y_j} \right) \left( \frac{\partial A_{mn}}{\partial y_i} + \frac{\partial A_{mn}}{\partial y_j} \right)
%& = \det(A)(A^{-1}_{ji}A^{-1}_{nm} - A^{-1}_{ni}A^{-1}_{jm})
\end{align}

or we can try:

\begin{align}
\frac{\partial^2 \det(A(\vy))}{\partial y_i \partial y_j}
& = \frac{\partial }{\partial \vy_j} \det(\Jy) \trace \left( \frac{\partial \Jy(\vy)}{\partial \vx} \right) \\
& = \det(\Jy) \frac{\partial }{\partial \vy_j} \trace \left( \frac{\partial \Jy(\vy)}{\partial \vx} \right) 
  + \trace \left( \frac{\partial \Jy(\vy)}{\partial \vx} \right) \frac{\partial }{\partial \vy_j} \det(\Jy) \\
\end{align}

or:

\begin{align}
\frac{\partial^2 \det(A(\vy))}{\partial y_i \partial y_j}
= \det(A) & \Big[ 
          \trace \left( A\inv \frac{\partial^2 A(\vy)}{\partial \vy_i \partial \vy_j } \right) \nonumber \\
          & + \trace \left( A\inv \frac{\partial A(\vy)}{\partial \vy_i} \right) 
            \trace \left( A\inv \frac{\partial A(\vy)}{\partial \vy_j} \right) \nonumber \\
           & + \trace \left( A\inv \frac{\partial A(\vy)}{\partial \vy_i} 
                          A\inv \frac{\partial A(\vy)}{\partial \vy_j} \right)
         \Big] \\
\frac{\partial^2 \det(\Jy(\vy))}{\partial y_i \partial y_j}
= \det(\Jy) & \Big[
          \trace \left( \Jy\inv \frac{\partial^2 \Jy(\vy)}{\partial \vy_i \partial \vy_j } \right) \nonumber \\
          & + \trace \left( \Jy\inv \frac{\partial \Jy(\vy)}{\partial \vy_i} \right) 
            \trace \left( \Jy\inv \frac{\partial \Jy(\vy)}{\partial \vy_j} \right) \nonumber \\
           & + \trace \left( \Jy\inv \frac{\partial \Jy(\vy)}{\partial \vy_i} 
                          \Jy\inv \frac{\partial \Jy(\vy)}{\partial \vy_j} \right)
         \Big] \\
= \det(\Jy) & \Big[
          \trace \left( \Jx \frac{\partial^2 \Jy(\vy)}{\partial \vy_i \partial \vy_j } \right) \nonumber \\
          & + \trace \left( \Jx \frac{\partial \Jy(\vy)}{\partial \vy_i} \right) 
            \trace \left( \Jx \frac{\partial \Jy(\vy)}{\partial \vy_j} \right) \nonumber \\
           & + \trace \left( \Jx \frac{\partial \Jy(\vy)}{\partial \vy_i} 
                          \Jx \frac{\partial \Jy(\vy)}{\partial \vy_j} \right)
         \Big] \\        
%
%\frac{\partial^2 \det(A(\vy))}{\partial A_{ij}\partial A_{mn}} \left( \frac{\partial A_{ij}}{\partial y_i} + \frac{\partial A_{ij}}{\partial y_j} \right) \left( \frac{\partial A_{mn}}{\partial y_i} + \frac{\partial A_{mn}}{\partial y_j} \right)
%& = \det(A)(A^{-1}_{ji}A^{-1}_{nm} - A^{-1}_{ni}A^{-1}_{jm})
\end{align}

\paragraph{Eigenvalues of Hessian}





\section{Random Matrix Theory}

``On the number of real eigenvalues of products of random matrices and an application to quantum entanglement''
\url{http://arxiv.org/pdf/1301.7601v1.pdf}

\paragraph{Stationarity} When characterizing deep \gp{}s, having a stationary kernel means that expectations of our function will be the same no matter which point we evaluate it at.  In other words, for any statistic $S(f(\vx))$, $\expect [ S( f(\vx) ] = \expect [ S( f(\vx') ] \forall \vx \forall \vx'$.  
%This is not true in general for deep \gplvm{}s.




%\begin{proposition}
\paragraph{The mean of a deep zero-mean \gp{} has mean zero.}
%The mean of a deep zero-mean \gp{} has mean zero.
%\end{proposition}
%
%\begin{proof}[Proof]
All functions are drawn \iid, so we can ignore all but the last transformation $f_L$.  Since $f_L$ is zero-mean, then $\forall \vx \forall d, \expectargs{\GP}{f_d(x)} = 0 $.
%\end{proof}


\subsection{Properties of the Jacobian}

\subsection{Gaussian Processes}

\subsection{Gaussian Process Latent Variable Model}

\begin{figure}
\centering
\includegraphics[width=0.8\columnwidth]{figures/gplvm_1d_draw_8} 
\caption{A draw from a Gaussian process latent variable model.  Bottom:  The latent datapoints $\vX$ are distributed according to a parametric base distribution (a Gaussian).  Top right:  A smooth function $f$ drawn from a Gaussian process prior is applied to obtain $\vY$ = $f(\vX)$.  Left:  The observed data $\vY$ is distributed according to a non-Gaussian density.}
\label{fig:gplvm_intro}
\end{figure}

The GP-LVM specifies a model wherein latent variables $\vX$ are warped by an unknown smooth, function $f$ to produce the observed data $\vY$.  The prior used over functions in the GP-LVM is the Gaussian process~\cite{rasmussen38gaussian}.

While not typically thought of as a density model, the GPLVM does define a nonparametric density over observations~\cite{nickisch2010gaussian}.   Figure \ref{fig:gplvm_intro} demonstrates how a Gaussian latent density, when warped by a random smooth function, can give rise to a non-Gaussian density in the observed space.

The dimension of the observed data ($D$) doesn't need to match the dimension of the latent space ($Q$).  When $Q$ is 2 or 3, the GP-LVM can also be used for visualization of high-dimensional data.  The mapping from $\vX$ to each dimension of the observed data is assumed to be independent, so the likelihood has a simple form which implicitly integrates over $f$:
%
\begin{align}
p(\vY | \vX,\bm{\theta})  = (2 \pi)^{-\frac{DN}{2}}  |\vK|^{-\frac{D}{2}} \exp \left( -\frac{1}{2} {\rm tr}( \vY^{\top} \vK^{-1} \vY) \right),
\label{eq:py_x}
\end{align}
where $\vK$ is the $N \times N$ covariance matrix defined 
by the kernel function $k(\vx_{n},\vx_{m})$,
and $\bm{\theta}$ is the kernel hyperparameter vector.
In this paper, we use an RBF kernel with an additive noise term:
\begin{align}
k(\vx_{n},\vx_{m}) &= \alpha \exp\left( - \frac{1}{2 \ell^2}(\vx_n - \vx_m)^{\top} (\vx_n - \vx_m) \right) + \delta_{nm} \beta^{-1}.
\end{align}




\paragraph{The density in the observed space of a deep density}

Let $\vy = \fdeep(\vx)$ be a random variable.
%
The change-of-variables formula is:
\begin{align}
p(f(\vx)) = \sum_{k=1}^{n(\vy_k = f(\vx_k))} \left| \frac{d}{dy} f^{-1}(\vy_{k}) \right| \cdot p_x(\vx_{k})
\end{align}
%
Assuming that $\fdeep(\vx)$ is one-to-one,
\begin{align}
p(f(\vx)) = p_x( \vx ) \left| \frac{  \partial f\inv(x) }{\partial x } \right|
\end{align}
%
We can use that
\begin{align}
\det(A\inv) = \frac{1}{\det(A)}
\end{align}
%
Assuming that $\fdeep(\vx)$ is one-to-one,
\begin{align}
p(f(\vx)) = p_x( \vx ) \left| J\inv(\vx) \right| = p_x( \vx ) \frac{1}{\left| J(\vx) \right|}
\end{align}

\paragraph{Eigenvalues of inverse}  We can try to prove things about the eigenvalues of $J(\vx)$.  However, in order to characterize the density of $p(\vy)$, we will need to analyse the eigenspectrum of $J\inv(\vx)$.  Helpfully, if $\lambda$ are the eigenvalues of $J$, then $\lambda \inv$ are the eigenvalues of $J\inv$.

\paragraph{Determinant in terms of Eigenvalues}  If $\lambda$ are the eigenvalues of J, then
\begin{align}
|J| = \prod_i \lambda_i
\end{align}







\section{Relation to Deep Neural Networks}

There are two reasons to think that the deep-GP prior is related to the inductive bias of deep neural networks.  First, the 

\paragraph{Weights don't change much during training}
James Martens says that the weights don't change much during training.  Perhaps we could make a plot showing the original weights versus the trained weights?

\paragraph{$L_2$ regularization of weights corresponds in a loose sense to independent Gaussian priors on the weights}
Which correspond to Gaussian processes (Neal)


\bibliography{verydeep}
\bibliographystyle{unsrt}

\end{document}
