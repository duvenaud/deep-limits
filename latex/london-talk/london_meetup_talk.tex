\input{header_beamer}
\usepackage{etex}
%\include{macros}
%\documentclass[usenames,dvipsnames]{beamer}
%\usepackage{beamerthemesplit}
%\usepackage{graphics}
%\usepackage{amsmath}
%\usepackage{rotating}
%\usepackage{array}
%\usepackage{nth}
\usepackage{xcolor}
\usepackage{textcomp}
\input{matlab_setup}

\usepackage{tabularx}
\usepackage{picins}
\usepackage{tikz}
\usepackage{changepage}
\usepackage{wasysym} % for smileys

\usetikzlibrary{shapes.geometric,arrows,chains,matrix,positioning,scopes,calc}
\tikzstyle{mybox} = [draw=white, rectangle]

\definecolor{camlightblue}{rgb}{0.601 , 0.8, 1}
\definecolor{camdarkblue}{rgb}{0, 0.203, 0.402}
\definecolor{camred}{rgb}{1, 0.203, 0}
\definecolor{camyellow}{rgb}{1, 0.8, 0}
\definecolor{lightblue}{rgb}{0, 0, 0.80}
\definecolor{white}{rgb}{1, 1, 1}
\definecolor{whiteblue}{rgb}{0.80, 0.80, 1}

\newcolumntype{x}[1]{>{\centering\arraybackslash\hspace{0pt}}m{#1}}
\newcommand{\tabbox}[1]{#1}

%%%%%%%%%%%%%%%%%%%%%%%%%%%%%%%%%%%%%%%%%%%
%
% Some look and feel definitions
%
%%%%%%%%%%%%%%%%%%%%%%%%%%%%%%%%%%%%%%%%%%%

\setlength{\columnsep}{0.03\textwidth}
\setlength{\columnseprule}{0.0018\textwidth}
\setlength{\parindent}{0.0cm}

%\include{macros}
\usepackage{preamble}
\hypersetup{colorlinks=true,citecolor=blue}
%\pdfmapfile{+sansmathaccent.map}

\title{Visualizing Priors on Deep Functions}


\author{
\hspace{-1cm}
\includegraphics[height=0.2\textwidth, trim=20mm 25mm 0mm 25mm, clip]{figures/david2}
\qquad\quad
\includegraphics[height=0.2\textwidth]{figures/rippel2}
\qquad\quad
\includegraphics[height=0.2\textwidth]{figures/adams}
\qquad\quad
\includegraphics[height=0.2\textwidth]{figures/zg2}
\hspace{-1cm}
\\
\hspace{-0.9cm}
David Duvenaud, Oren Rippel, Ryan Adams, Zoubin Ghahramani
\hspace{-0.9cm}
}

\institute{
%\includegraphics[width=0.4\textwidth]{figures/spiral_main}
}
%\date{}


\begin{document}

\setbeamertemplate{background canvas}{\begin{tikzpicture}\node[opacity=.1]{\includegraphics [width=\paperwidth]{figures/map_connected/latent_coord_map_layer_40}};\end{tikzpicture}}



\frame[plain] {
\titlepage
}

\setbeamertemplate{background canvas}{}



\frame[plain]{
\frametitle{Motivation 1: Deep Learning is Cool}
\begin{itemize}
	\item Don't you wish we were doing it?
	\item Zoubin (2011) ``Do you guys ever wonder if this lab focuses too much on Gaussian processes?  Like maybe we're going to miss the next big thing, like maybe, say, deep learning''
	\item But - GPs are just neural nets, we can make them deep!
\end{itemize}
}

\frame[plain]{
\frametitle{Motivation 2: Regularizing Nets}
\begin{itemize}
	\item Neural nets are getting larger
	\item How to regularize billions of parameters?
	\item Closely related to constructing priors
	\item Priors are easy to analyze - just sample from the prior and look and what sorts of things you get!
%	\item \color{blue} Can we write a useful paper without doing any experiments?
	\item \color{blue} Can we suggest new regularization schemes or network architectures?
\end{itemize}
}



\frame[plain]{
\frametitle{Outline}
\begin{itemize}
	\item Relation between GPs and neural nets
	\item Two ways to deepness:
	\begin{itemize}
		\item Deep kernels
		\item Deep GPs
	\end{itemize}
	\item What kind of prior on functions do we want?
	\begin{itemize}
		\item problems with lots of indepenent layers
		\item a simple fix
	\end{itemize}
	\item Dropout for GPs
	\begin{itemize}
		\item Dropping out features
		\item Dropping out inputs
	\end{itemize}	
\end{itemize}
}


\newsavebox\unistrain
\begin{lrbox}{\unistrain}
\hspace{0.1cm}
  \begin{minipage}{0.53\textwidth}
A weighted sum of features,
    \begin{align*}
%\Phi(\vx) & = \vh^{(1)}(\vx) = \sigma \left( \vb^{(1)} + \vW^{(1)}\vx \right) \\
%f(\vx) & = \vV^{(1)} \sigma \left( \vb^{(1)} + \vW^{(1)} \vh^{(1)}(\vx) \right)  = \vV^{(1)} \vh^{(1)}(\vx) \\
f(\vx) & %= \frac{1}{K}{\mathbf \alpha}\tra \Phi(\vx) 
= \frac{1}{K} \sum_{i=1}^K \alpha_i \phi_i(\vx)
    \end{align*} 
with any weight distribution,
    \begin{align*}
%\Phi(\vx) & = \left[ \phi_1(\vx), \dots, \phi_K(\vx) \right]\tra \\
\expectargs{}{\alpha_i} & = 0, \quad \varianceargs{}{\alpha_i} = \sigma^2, \quad \iid
    \end{align*} 
by CLT, gives a GP as $K \to \infty$!
        \begin{align*}
%\lim_{K \to \infty} & \left[ f(\vx), f(\vx') \right ]\tra 
\cov \left[ \! \begin{array}{c} f(\vx) \\ f(\vx') \end{array} \! \right] \to \frac{\sigma^2}{K}\sum_{i=1}^K \phi_i(\vx)\phi_i(\vx')
    \end{align*} 
  \end{minipage}
\end{lrbox}


\def\layersep{2cm}
\def\nodesep{1.5cm}
\def\nodesize{1cm}

\newcommand{\numdims}[0]{3}
\newcommand{\numouts}[0]{1}
\newcommand{\numhidden}[0]{4}
\newcommand{\upnodedist}[0]{1cm}
\newcommand{\bardist}[0]{\hspace{-0.2cm}}

\frame[plain]{
\frametitle{GPs as Neural Nets}

\vspace{0.5cm}
\begin{tabular}{c|c}
\hspace{-1.25cm}
\begin{minipage}{0.54\textwidth}
\begin{tikzpicture}[shorten >=1pt,->,draw=black!50, node distance=\layersep]
    \tikzstyle{every pin edge}=[<-,shorten <=1pt]
    \tikzstyle{neuron}=[circle,fill=black!25,minimum size=17pt,inner sep=0pt]
    \tikzstyle{input neuron}=[neuron, fill=green!30];
    \tikzstyle{output neuron}=[neuron, fill=red!30];
    \tikzstyle{hidden neuron}=[neuron, fill=blue!30];
    \tikzstyle{annot} = [text width=4em, text centered]

    % Draw the input layer nodes
    \foreach \name / \y in {1,...,\numdims}
    % This is the same as writing \foreach \name / \y in {1/1,2/2,3/3,4/4}
        \node[input neuron, minimum size=\nodesize
        %, pin=left:Input \#\y
        ] (I-\name) at (0,-\nodesep*\y) {$x_\y$};

    % Draw the hidden layer nodes
    \foreach \name / \y in {1,...,\numhidden}
        \path[yshift=0.5cm]
            node[hidden neuron, minimum size=\nodesize] (H-\name) at (\layersep,-\nodesep*\y) {$\phi_\y(\vx)$};

    % Draw the output layer node
    \foreach \name / \y in {1,...,\numouts}
    	\node[output neuron, minimum size=\nodesize
    	%,pin={[pin edge={->}]right:Output }
    	] (O-\name) at (2*\layersep,-\nodesep*2) {$f(x)$};

    % Connect every node in the input layer with every node in the
    % hidden layer.
    \foreach \source in {1,...,\numdims}
        \foreach \dest in {1,...,\numhidden}
            \path (I-\source) edge (H-\dest);

    % Connect every node in the hidden layer with the output layer
    \foreach \source in {1,...,\numhidden}
        \foreach \dest in {1,...,\numouts}
    	    \path (H-\source) edge (O-\dest);

    % Annotate the layers
    \node[annot,above of=I-1, node distance=\upnodedist] {Inputs};
    \node[annot,above of=H-1, node distance=\upnodedist] {Hidden};
    \node[annot,above of=O-1, node distance=\upnodedist] {Output};
\end{tikzpicture} 
\end{minipage}
&
\usebox{\unistrain}
  \end{tabular}
}



\frame[plain]{
\frametitle{Kernel learning as feature learning}
\begin{itemize}
	\item GPs have fixed features, integrate out feature weights.
%	\item Neural nets with tractable marginal likelihood!	
	\item Mapping between kernels and features:  $k(\vx,\vx') = \Phi(\vx)\tra \Phi(\vx')$.
%	\vspace{\baselineskip}
	\item Any PSD kernel can be written as inner product of features. (Mercer's Theorem)
	\item Kernel learning = feature learning
	\vspace{\baselineskip}
	\item What if we make the GP nueral network deep?
%	\item example: periodic kernel $k_{per}(x,x') = \exp( - \sin^2(x - x') )$ is equiavelent to $k_{se}(\sin(x), \cos(x), \sin(x'), \cos(x')$.
%	\vspace{\baselineskip}
%	\item What can we do with feature compositions?
\end{itemize}
}





\newsavebox\deepkernels
\begin{lrbox}{\deepkernels}
  \begin{minipage}{0.4\textwidth}
%	({\color{blue}Cho, 2012}) built kernels from multiple layers of feature mappings.
Now our model is:
    \begin{align*}
%\Phi(\vx) & = \vh^{(1)}(\vx) = \sigma \left( \vb^{(1)} + \vW^{(1)}\vx \right) \\
%f(\vx) & = \vV^{(1)} \sigma \left( \vb^{(1)} + \vW^{(1)} \vh^{(1)}(\vx) \right)  = \vV^{(1)} \vh^{(1)}(\vx) \\
 %= \frac{1}{K}{\mathbf \alpha}\tra \Phi(\vx) 
f(\vx) = & \frac{1}{K} \sum_{i=1}^K \alpha_i \phi_i \left( \Phi^{(1)}(\vx) \right) \\
= & \bm{\alpha}\tra \Phi^{(2)} \left( \Phi^{(1)}(\vx) \right)
    \end{align*} 
	Instead of 
	$$k_1(\vx, \vx') = \Phi^{(1)}(\vx) \tra \Phi^{(1)}(\vx'),$$
	we have ``deep kernel'':
	\begin{align*}
	& k_2(\vx, \vx') \\ 
	& = \Phi^{(2)}(\Phi^{(1)}(\vx)) \tra \Phi^{(2)}(\Phi^{(1)}(\vx'))
	\end{align*}
  \end{minipage}
\end{lrbox}


\frame[plain]{
\frametitle{Deep nets, deep kernels}
\begin{tabular}{c|c}
\begin{minipage}{0.535\textwidth}
\begin{tikzpicture}[shorten >=1pt,->,draw=black!50, node distance=\layersep]
\hspace{-1.4cm}
    \tikzstyle{every pin edge}=[<-,shorten <=1pt]
    \tikzstyle{neuron}=[circle,fill=black!25,minimum size=17pt,inner sep=0pt]
    \tikzstyle{input neuron}=[neuron, fill=green!30];
    \tikzstyle{output neuron}=[neuron, fill=red!30];
    \tikzstyle{hidden neuron}=[neuron, fill=blue!30];
    \tikzstyle{annot} = [text width=4em, text centered]

    % Draw the input layer nodes
    \foreach \name / \y in {1,...,\numdims}
    % This is the same as writing \foreach \name / \y in {1/1,2/2,3/3,4/4}
        \node[input neuron, minimum size=\nodesize
        %, pin=left:Input \#\y
        ] (I-\name) at (0,-\nodesep*\y) {$x_\y$};

    % Draw the hidden layer nodes
    \foreach \name / \y in {1,...,\numhidden}
        \path[yshift=0.5cm]
            node[hidden neuron, minimum size=\nodesize] (H-\name) at (\layersep,-\nodesep*\y) {$\phi^{(1)}_\y$};

    % Draw the hidden layer nodes
    \foreach \name / \y in {1,...,\numhidden}
        \path[yshift=0.5cm]
            node[hidden neuron, minimum size=\nodesize] (H2-\name) at (2*\layersep,-\nodesep*\y) {$\phi^{(2)}_\y$};

    % Draw the output layer node
    \foreach \name / \y in {1,...,\numouts}
    	\node[output neuron, minimum size=\nodesize
    	%,pin={[pin edge={->}]right:Output }
    	] (O-\name) at (2.8*\layersep,-\nodesep*2) {$f(\vx)$};

    % Connect every node in the input layer with every node in the
    % hidden layer.
    \foreach \source in {1,...,\numdims}
        \foreach \dest in {1,...,\numhidden}
            \path (I-\source) edge (H-\dest);
            
    \foreach \source in {1,...,\numhidden}
        \foreach \dest in {1,...,\numhidden}
            \path (H-\source) edge (H2-\dest);            

    % Connect every node in the hidden layer with the output layer
    \foreach \source in {1,...,\numhidden}
        \foreach \dest in {1,...,\numouts}
    	    \path (H2-\source) edge (O-\dest);

    % Annotate the layers
    \node[annot,above of=I-1, node distance=\upnodedist] {Inputs};
    \node[annot,above of=H-1, node distance=\upnodedist] {Hidden};
    \node[annot,above of=H2-1, node distance=\upnodedist] {Hidden};
    \node[annot,above of=O-1, node distance=\upnodedist] {Output};
\end{tikzpicture}
\end{minipage}
&
\usebox{\deepkernels}
  \end{tabular}
}



\newsavebox\deepkernelstwo
\begin{lrbox}{\deepkernelstwo}
  \begin{minipage}{0.4\textwidth}
%	({\color{blue}Cho, 2012}) built kernels from multiple layers of feature mappings.
Now our model is:
    \begin{align*}
\Phi^{1}(x) = \left[ \sin(x), \cos(x) \right]
    \end{align*} 
	we have ``deep kernel'':
	\begin{align*}
	& k_2(\vx, \vx') \\ 
	& = \exp(-\frac{1}{2} \left( \Phi^{1}(\vx)) - \Phi^{1}(\vx') \right)
	\end{align*}
  \end{minipage}
\end{lrbox}

\newcommand{\numhiddenper}[0]{2}

\frame[plain]{
\frametitle{Example deep kernel: Periodic}
\begin{tabular}{c|c}
\begin{minipage}{0.535\textwidth}
\begin{tikzpicture}[shorten >=1pt,->,draw=black!50, node distance=\layersep]
\hspace{-1.4cm}
    \tikzstyle{every pin edge}=[<-,shorten <=1pt]
    \tikzstyle{neuron}=[circle,fill=black!25,minimum size=17pt,inner sep=0pt]
    \tikzstyle{input neuron}=[neuron, fill=green!30];
    \tikzstyle{output neuron}=[neuron, fill=red!30];
    \tikzstyle{hidden neuron}=[neuron, fill=blue!30];
    \tikzstyle{annot} = [text width=4em, text centered]

    % Draw the input layer nodes
    \foreach \name / \y in {1,...,1}
    % This is the same as writing \foreach \name / \y in {1/1,2/2,3/3,4/4}
        \node[input neuron, minimum size=\nodesize
        %, pin=left:Input \#\y
        ] (I-\name) at (0,-\nodesep*2) {$x$};

    % Draw the hidden layer nodes
%    \foreach \name / \y in {1,...,\numhiddenper}
    \path[yshift=0.5cm]
    node[hidden neuron, minimum size=\nodesize] (H-1) at (\layersep,-\nodesep*2) {$\sin(x)$};
   \path[yshift=0.5cm]
    node[hidden neuron, minimum size=\nodesize] (H-2) at (\layersep,-\nodesep*3) {$\cos(x)$};    

    % Draw the hidden layer nodes
    \foreach \name / \y in {1,...,\numhidden}
        \path[yshift=0.5cm]
            node[hidden neuron, minimum size=\nodesize] (H2-\name) at (2*\layersep,-\nodesep*\y) {$\phi^{(2)}_\y$};

    % Draw the output layer node
    \foreach \name / \y in {1,...,\numouts}
    	\node[output neuron, minimum size=\nodesize
    	%,pin={[pin edge={->}]right:Output }
    	] (O-\name) at (2.8*\layersep,-\nodesep*2) {$f(\vx)$};

    % Connect every node in the input layer with every node in the
    % hidden layer.
    \foreach \source in {1,...,1}
        \foreach \dest in {1,...,\numhiddenper}
            \path (I-\source) edge (H-\dest);
            
    \foreach \source in {1,...,\numhiddenper}
        \foreach \dest in {1,...,\numhidden}
            \path (H-\source) edge (H2-\dest);            

    % Connect every node in the hidden layer with the output layer
    \foreach \source in {1,...,\numhidden}
        \foreach \dest in {1,...,\numouts}
    	    \path (H2-\source) edge (O-\dest);

    % Annotate the layers
    \node[annot,above of=I-1, node distance=\upnodedist] {Inputs};
    \node[annot,above of=H-1, node distance=\upnodedist] {Hidden};
    \node[annot,above of=H2-1, node distance=\upnodedist] {Hidden};
    \node[annot,above of=O-1, node distance=\upnodedist] {Output};
\end{tikzpicture}
\end{minipage}
&
\usebox{\deepkernelstwo}
  \end{tabular}
}





\frame[plain]{
\frametitle{Deep Kernels}
\begin{itemize}
	\item ({\color{blue}Cho, 2012}) built kernels by composing feature mappings.
	\item Composing any kernel $k_1$ with a squared-exp kernel (SE):
%
\begin{align*}
%k_1(\vx, \vx') & = \exp \left( -\frac{1}{2} ||\vx - \vx'||_2^2 \right) \\
& k_2(\vx, \vx') = \\
& = \left( \Phi^{SE} \left(\Phi^{1}(\vx) \right) \right) \tra \Phi^{SE} \left( \Phi^{1}(\vx') \right) \\
& = \exp \left( -\frac{1}{2} || \Phi^{1}(\vx) - \Phi^{1}(\vx')||_2^2 \right) \nonumber\\
%k_{n+1}(\vx, \vx') 
%& = \exp \left( -\frac{1}{2} \sum_i \left[ \phi_n^{(i)}(\vx) - \phi_n^{(i)}(\vx') \right]^2 \right) \\
& = \exp\left ( -\frac{1}{2} \left[ \Phi^{1}(\vx) \tra \Phi^{1}(\vx) - 2 \Phi^{1}(\vx) \tra \Phi^{1}(\vx') + \Phi^{1}(\vx') \tra \Phi^{1}(\vx') \right] \right) \\
%k_2(\vx, \vx') & = \exp \left( -\frac{1}{2} \left[ \sum_i \phi_i(\vx)^2 - 2 \sum_i \phi_i(\vx) \phi_i(\vx') + \sum_i \phi_i(\vx')^2 \right] \right) \\
%k_{n+1}(\vx, \vx') 
& = \exp \left( -\frac{1}{2} \left[ k_1(\vx, \vx) - 2 k_1(\vx, \vx') + k_1(\vx', \vx') \right] \right)
%k_{n+1}(\vx, \vx') 
%& = \exp \left( k_1(\vx, \vx') - 1 \right) \qquad \textnormal{(if $k_1(\vx, \vx) = 1$)} \nonumber
\end{align*}
%
\item A closed form\dots let's do it again!
\end{itemize}
}




%\frame[plain]{
%\frametitle{Deep Kernels}
%\includegraphics[height=0.4\textwidth]{figures/inception}
%}


\frame[plain]{
\frametitle{We need to go deeper}
%\begin{tabular}{c|c}
%\begin{minipage}{0.535\textwidth}
\begin{tikzpicture}[shorten >=1pt,->,draw=black!50, node distance=\layersep]
%\hspace{-1.4cm}
    \tikzstyle{every pin edge}=[<-,shorten <=1pt]
    \tikzstyle{neuron}=[circle,fill=black!25,minimum size=17pt,inner sep=0pt]
    \tikzstyle{input neuron}=[neuron, fill=green!30];
    \tikzstyle{output neuron}=[neuron, fill=red!30];
    \tikzstyle{hidden neuron}=[neuron, fill=blue!30];
    \tikzstyle{annot} = [text width=4em, text centered]

    % Draw the input layer nodes
    \foreach \name / \y in {1,...,\numdims}
    % This is the same as writing \foreach \name / \y in {1/1,2/2,3/3,4/4}
        \node[input neuron, minimum size=\nodesize
        %, pin=left:Input \#\y
        ] (I-\name) at (0,-\nodesep*\y) {$x_\y$};

    % Draw the hidden layer nodes
    \foreach \name / \y in {1,...,\numhidden}
        \path[yshift=0.5cm]
            node[hidden neuron, minimum size=\nodesize] (H-\name) at (\layersep,-\nodesep*\y) {$\phi^{(1)}_\y$};

    % Draw the hidden layer nodes
    \foreach \name / \y in {1,...,\numhidden}
        \path[yshift=0.5cm]
            node[hidden neuron, minimum size=\nodesize] (H2-\name) at (2*\layersep,-\nodesep*\y) {$\phi^{(2)}_\y$};
            
    % Draw the hidden layer nodes
    \foreach \name / \y in {1,...,\numhidden}
        \path[yshift=0.5cm]
            node[hidden neuron, minimum size=\nodesize] (H3-\name) at (3*\layersep,-\nodesep*\y) {$\phi^{(3)}_\y$};       
            
    % Draw the hidden layer nodes
    \foreach \name / \y in {1,...,\numhidden}
        \path[yshift=0.5cm]
            node[hidden neuron, minimum size=\nodesize] (H4-\name) at (4*\layersep,-\nodesep*\y) {$\phi^{(4)}_\y$};                    

    % Draw the output layer node
    \foreach \name / \y in {1,...,\numouts}
    	\node[output neuron, minimum size=\nodesize
    	%,pin={[pin edge={->}]right:Output }
    	] (O-\name) at (5*\layersep,-\nodesep*2) {$f(\vx)$};

    % Connect every node in the input layer with every node in the
    % hidden layer.
    \foreach \source in {1,...,\numdims}
        \foreach \dest in {1,...,\numhidden}
            \path (I-\source) edge (H-\dest);
            
    \foreach \source in {1,...,\numhidden}
        \foreach \dest in {1,...,\numhidden}
            \path (H-\source) edge (H2-\dest);  
            
    \foreach \source in {1,...,\numhidden}
        \foreach \dest in {1,...,\numhidden}
            \path (H2-\source) edge (H3-\dest);  
            
    \foreach \source in {1,...,\numhidden}
        \foreach \dest in {1,...,\numhidden}
            \path (H3-\source) edge (H4-\dest);                                    

    % Connect every node in the hidden layer with the output layer
    \foreach \source in {1,...,\numhidden}
        \foreach \dest in {1,...,\numouts}
    	    \path (H4-\source) edge (O-\dest);

    % Annotate the layers
%    \node[annot,above of=I-1, node distance=\upnodedist] {Inputs};
%    \node[annot,above of=H-1, node distance=\upnodedist] {Hidden};
%    \node[annot,above of=H2-1, node distance=\upnodedist] {Hidden};
%    \node[annot,above of=O-1, node distance=\upnodedist] {Output};
\end{tikzpicture}
%\end{minipage}
%&
%\usebox{\deepkernels}
%  \end{tabular}
}





\frame[plain]{
\frametitle{Infinitely Deep Kernels}
\begin{itemize}
	\item For SE kernel, $k_{L+1}(\vx, \vx') = \exp \left( k_L(\vx, \vx') - 1 \right)$.
	\item What is the limit of composing SE features?
\end{itemize}
\centering
\begin{tabular}{cc}
\includegraphics[width=0.55\columnwidth, clip, trim = 0cm 0cm 0cm 0.61cm]{figures/deep_kernel} &
\hspace{-1cm}\includegraphics[width=0.55\columnwidth, clip, trim = 0cm 0cm 0cm 0.61cm]{figures/deep_kernel_draws} \\
Kernel & Draws from GP prior
\end{tabular}

\begin{itemize}
	\item $k_\infty(\vx, \vx') = 1$ everywhere.  \frownie
\end{itemize}
}



\frame[plain]{
\frametitle{A simple fix...}
\begin{itemize}
	\item Following a suggestion from {\color{blue!80} Neal (1995)}, we 
connect the inputs $\vx$ to each layer:

\vspace{0.5cm}

\def\nodeseptwo{4cm}
\def\nodesize{.45cm}
\def\numhiddentwo{4}


%\newlength{\arrowsize}  
%\pgfarrowsdeclare{biggertip}{biggertip}{  
%  \setlength{\arrowsize}{1pt}  
%  \addtolength{\arrowsize}{.5\pgflinewidth}  
%  \pgfarrowsrightextend{0}  
%  \pgfarrowsleftextend{-5\arrowsize}  
%}{  
%  \setlength{\arrowsize}{0.4pt}  
%  \addtolength{\arrowsize}{.5\pgflinewidth}  
%  \pgfpathmoveto{\pgfpoint{-5\arrowsize}{4\arrowsize}}  
%  \pgfpathlineto{\pgfpointorigin}  
%  \pgfpathlineto{\pgfpoint{-5\arrowsize}{-4\arrowsize}}  
%  \pgfusepathqstroke  
%} 

\begin{tabular}{ccc}
Standard deep net architecture & & Input-connected architecture \\
\bardist
\begin{tikzpicture}[draw=black]

    \tikzstyle{neuron}=[circle,minimum size=17pt, draw = black, fill = white, thick]
    \tikzstyle{input neuron}=[neuron, fill=green!50];
    \tikzstyle{output neuron}=[neuron, fill=red!50];
    \tikzstyle{hidden neuron}=[neuron, fill=blue!50];
    \tikzstyle{pile} =[ultra thick, ->, >=stealth', shorten <= 0.6cm, shorten >= 0.6cm, -biggertip, line width = 2pt];

    % Define the input layer node
    \coordinate (I) at (0, 0);


    % Define the hidden layer nodes
    \foreach \name / \y in {1,...,\numhiddentwo}
    {
        \coordinate (H-\name) at (\nodeseptwo*\y, 0);
    }

    % Connect every node            
    \foreach \name in {1,...,\numhiddentwo}
    {
	 \path[pile] (I) edge (H-\name) {};
         %\path[pile] (I) edge [bend left] (H-\name) {};
    }

    \draw (I) node[neuron, ultra thick] {};
    \draw (I) node[below = 0.5cm]  {$\vx$};

    % Draw the hidden layer nodes
    \foreach \name / \y in {1,...,\numhiddentwo}
    {
	\draw (H-\name) node[neuron, ultra thick]  {};
        \draw (H-\name) node[below = 0.34cm] {$\vf^{(\y)}$};
    }
\end{tikzpicture} & \hspace{2cm} &
\bardist
\begin{tikzpicture}[draw=black]
    \tikzstyle{neuron}=[circle,minimum size=17pt, draw = black, fill = white, thick]
    \tikzstyle{input neuron}=[neuron, fill=green!50];
    \tikzstyle{output neuron}=[neuron, fill=red!50];
    \tikzstyle{hidden neuron}=[neuron, fill=blue!50];
    \tikzstyle{pile} =[ultra thick, ->, >=stealth', shorten <= 0.6cm, shorten >= 0.6cm, -biggertip, line width = 2pt];

    % Define the input layer node
    \coordinate (I) at (0, 0);


    % Define the hidden layer nodes
    \foreach \name / \y in {1,...,\numhiddentwo}
    {
        \coordinate (H-\name) at (\nodeseptwo*\y, 0);
    }

    % Connect every node            
    \foreach \name in {1,...,\numhiddentwo}
    {
	 \path[pile] (I) edge (H-\name) {};
         \path[pile] (I) edge [bend left] (H-\name) {};
    }

    \draw (I) node[neuron, ultra thick] {};
    \draw (I) node[below = 0.5cm]  {$\vx$};

    % Draw the hidden layer nodes
    \foreach \name / \y in {1,...,\numhiddentwo}
    {
	\draw (H-\name) node[neuron, ultra thick]  {};
        \draw (H-\name) node[below = 0.34cm] {$\vf^{(\y)}$};
    }
\end{tikzpicture}
\end{tabular}

\end{itemize}
}



\frame[plain]{
\frametitle{A simple fix...}
\begin{itemize}
	\item Following a suggestion from {\color{blue!80} Neal (1995)}, we 
connect the inputs $\vx$ to each layer:
%
\begin{align*}
%k_1(\vx, \vx') & = \exp \left( -\frac{1}{2} ||\vx - \vx'||_2^2 \right) \\
& k_{L+1}(\vx, \vx') = \nonumber \\
& = \exp \left( -\frac{1}{2} \left|\left| \left[ \! \begin{array}{c} \Phi^L(\vx) \\ {\color{red} \vx} \end{array} \! \right]  - \left[ \! \begin{array}{c} \Phi^L(\vx') \\ {\color{red} \vx'} \end{array} \! \right] \right| \right|_2^2 \right) \nonumber \\
%k_{n+1}(\vx, \vx') 
%& = \exp \left( -\frac{1}{2} \sum_i \left[ \phi_i(\vx) - \phi_i(\vx') \right]^2 -\frac{1}{2} || \vx - \vx' ||_2^2 \right) \\
%k_{n+1}(\vx, \vx') & = \exp\left ( -\frac{1}{2} \sum_i \left[ \phi_i(\vx)^2 - 2 \phi_i(\vx) \phi_i(\vx') + \phi_i(\vx')^2 \right]  -\frac{1}{2} || \vx - \vx' ||_2^2 \right) \\
%k_2(\vx, \vx') & = \exp \left( -\frac{1}{2} \left[ \sum_i \phi_i(\vx)^2 - 2 \sum_i \phi_i(\vx) \phi_i(\vx') + \sum_i \phi_i(\vx')^2 \right] \right) \\
%k_2(\vx, \vx') & = \exp \left( -\frac{1}{2} \left[ k_1(\vx, \vx) - 2 k_1(\vx, \vx') + k_1(\vx', \vx') \right] \right) \\
%k_{n+1}(\vx, \vx') 
& = \exp \left( -\frac{1}{2} \left[ k_L(\vx, \vx) - 2 k_L(\vx, \vx') + k_L(\vx', \vx') \right] {\color{red} -\frac{1}{2} || \vx - \vx' ||_2^2} \right)
\end{align*}
%\item What is the eventual limit?
\end{itemize}
}


\frame[plain]{
\frametitle{Infinitely Deep Kernels}
\begin{itemize}
	\item What is the limit of compositions of input-connected SE features?
	\item $k_{L+1}(\vx, \vx') = \exp \left( k_L(\vx, \vx') - 1 -\frac{1}{2} || \vx - \vx' ||_2^2 \right)$.	
\end{itemize}
\centering
\begin{tabular}{cc}
\includegraphics[width=0.56\columnwidth, clip, trim = 0cm 0cm 0cm 0.61cm]{figures/deep_kernel_connected} &
\hspace{-1cm}\includegraphics[width=0.55\columnwidth, clip, trim = 0cm 0cm 0cm 0.61cm]{figures/deep_kernel_connected_draws} \\
Kernels & Draws from GP priors
\end{tabular}

\begin{itemize}
	\item Like an Ornstein-Uhlenbeck process with skinny tails
	\item Samples are non-differentiable (fractal).
\end{itemize}
}


\frame[plain]{
\frametitle{What went wrong?}
\begin{itemize}
	\item Fixed feature mapping, unlikely to be useful for anything
%	\item only one type of structure repeated
%	\item not capturing invariances?
%	\item not throwing away unnecessary information
	\item power of neural nets comes from learning a custom representation
	\item Need to search over feature mappings!
	\item Can try to learn kernels, or even better, integrate over feature mappings
\end{itemize}
}





%\newcommand{\numdims}[0]{3}
%\newcommand{\numouts}[0]{1}
%\newcommand{\numhidden}[0]{4}
%\newcommand{\upnodedist}[0]{1cm}
%\newcommand{\bardist}[0]{\hspace{-0.2cm}}

\frame[plain]{
\frametitle{Deep Gaussian Processes}
\begin{tikzpicture}[shorten >=1pt,->,draw=black!50, node distance=\layersep]
    \tikzstyle{every pin edge}=[<-,shorten <=1pt]
    \tikzstyle{neuron}=[circle,fill=black!25,minimum size=17pt,inner sep=0pt]
    \tikzstyle{input neuron}=[neuron, fill=green!50];
    \tikzstyle{output neuron}=[neuron, fill=red!50];
    \tikzstyle{hidden neuron}=[neuron, fill=blue!50];
    \tikzstyle{annot} = [text width=4em, text centered]

    % Draw the input layer nodes
    \foreach \name / \y in {1,...,\numdims}
    % This is the same as writing \foreach \name / \y in {1/1,2/2,3/3,4/4}
        \node[input neuron, minimum size=\nodesize
        %, pin=left:Input \#\y
        ] (I-\name) at (0,-\nodesep*\y) {$x_\y$};

    % Draw the hidden layer nodes
    \foreach \name / \y in {1,...,\numhidden}
        \path[yshift=0.5cm]
            node[hidden neuron, minimum size=\nodesize] (H-\name) at (\layersep,-\nodesep*\y)  {$\phi^{(1)}_\y$};

    % Draw the hidden layer nodes
    \foreach \name / \y in {1,...,\numhidden}
        \path[yshift=0.5cm]
            node[hidden neuron, minimum size=\nodesize] (H2-\name) at (3*\layersep,-\nodesep*\y)  {$\phi^{(2)}_\y$};

    % Draw the output layer node
    \foreach \name / \y in {1,...,\numdims}
    	\node[output neuron, minimum size=\nodesize
    	%,pin={[pin edge={->}]right:Output }
    	] (O1-\name) at (2*\layersep,-\nodesep*\y) {$f^{(1)}_\y$};

    % Draw the output layer node
    \foreach \name / \y in {1,...,\numdims}
    	\node[output neuron, minimum size=\nodesize
    	%,pin={[pin edge={->}]right:Output }
    	] (O2-\name) at (4*\layersep,-\nodesep*\y) {$f^{(1:2)}_\y$};

    % Connect every node in the input layer with every node in the
    % hidden layer.
    \foreach \source in {1,...,\numdims}
        \foreach \dest in {1,...,\numhidden}
            \path (I-\source) edge (H-\dest);
            
    \foreach \source in {1,...,\numhidden}
        \foreach \dest in {1,...,\numdims}
            \path (H-\source) edge (O1-\dest);         
            
    \foreach \source in {1,...,\numdims}
        \foreach \dest in {1,...,\numhidden}
            \path (O1-\source) edge (H2-\dest);                

    % Connect every node in the hidden layer with the output layer
    \foreach \source in {1,...,\numhidden}
        \foreach \dest in {1,...,\numdims}
    	    \path (H2-\source) edge (O2-\dest);

    % Annotate the layers
    \node[annot,above of=I-1, node distance=\upnodedist] {Inputs};
    \node[annot,above of=H-1, node distance=\upnodedist] {Hidden};
    \node[annot,above of=O1-1, node distance=\upnodedist] {$\vf^{(1)}(\vx)$};
    \node[annot,above of=H2-1, node distance=\upnodedist] {Hidden};
    \node[annot,above of=O2-1, node distance=\upnodedist] {$\vf^{(1:2)}(\vx)$};

\end{tikzpicture}
}


\def\ie{i.e.\ }
\def\eg{e.g.\ }
\def\iid{i.i.d.\ }
%\def\simiid{\sim_{\mbox{\tiny iid}}}
\def\simiid{\overset{\mbox{\tiny iid}}{\sim}}
\def\simind{\overset{\mbox{\tiny \textnormal{ind}}}{\sim}}
\def\eqdist{\stackrel{\mbox{\tiny d}}{=}}
\newcommand{\distas}[1]{\mathbin{\overset{#1}{\kern\z@\sim}}}
%\newcommand{\vf}{\vect{f}}
\newcommand{\GPt}[2]{\mathcal{GP}\!\left(#1,#2\right)}

\frame[plain]{
\frametitle{Deep Gaussian Processes}
\begin{itemize}
	\item a prior over compositions of functions:
	\begin{align}
\vf^{(1:L)}(\vx) = \vf^{(L)}(\vf^{(L-1)}(\dots \vf^{(2)}(\vf^{(1)}(\vx)) \dots))
\end{align}
%
where each $\vf_d^{(\ell)} \simind \GPt{0}{k^\ell_d(\vx, \vx')}$. 
	\item Analogous to neural nets, where each neuron's activation function is an independent draw from a GP.
	\item inference is really hard.	
	\item maybe we can learn something just from looking at draws?
\end{itemize}
}

\newcommand{\onedsamplepic}[1]{
\includegraphics[width=0.7\columnwidth]{figures/1d_samples/latent_seed_0_1d_large/layer-#1}}

\frame[plain]{
\frametitle{Deep Gaussian Processes}
\begin{itemize}
	\item Draws from one-dimensional deep GPs:
	\vspace{\baselineskip}
	\only<1>{\onedsamplepic{1}}
	\only<2>{\onedsamplepic{2}}
	\only<3>{\onedsamplepic{3}}
	\only<4>{\onedsamplepic{4}}
	\only<5>{\onedsamplepic{5}}
	\only<6>{\onedsamplepic{6}}
	\only<7>{\onedsamplepic{7}}
	\only<8>{\onedsamplepic{8}}
	\only<9>{\onedsamplepic{9}}
	\only<10>{\onedsamplepic{10}}
\end{itemize}
}


\newcommand{\gpdrawbox}[1]{
\setlength\fboxsep{0pt}
\fbox{
\includegraphics[width=0.67\columnwidth]{figures/deep_draws/deep_gp_sample_layer_#1}
}}

\frame[plain]{
\frametitle{Deep Gaussian Processes}
\begin{itemize}
	\item 2D to 2D warpings of a Gaussian density:
	\vspace{\baselineskip}
	\only<1>{\gpdrawbox{1}}
	\only<2>{\gpdrawbox{2}}
	\only<3>{\gpdrawbox{3}}
	\only<4>{\gpdrawbox{4}}
	\only<5>{\gpdrawbox{5}}
	\only<6>{\gpdrawbox{6}}
\end{itemize}
}





\newcommand{\mappic}[1]{
%\includegraphics[width=0.6\columnwidth]{figures/map/latent_coord_map_layer_#1}
\includegraphics[width=0.67\columnwidth]{figures/seed-0-map/latent_coord_map_layer_#1}
} 
\newcommand{\mappiccon}[1]{
\includegraphics[width=0.67\columnwidth]{figures/seed-0-map-connected/latent_coord_map_layer_#1}
%\includegraphics[width=0.6\columnwidth]{figures/map_connected/latent_coord_map_layer_#1}
}


\frame[plain]{
\frametitle{Deep Gaussian Processes}
\begin{itemize}
	\item Showing the x that gave rise to a particular y
	\item (i.e. decision boundaries) \\
	\vspace{\baselineskip}
	\only<1>{\mappic{0} \quad No warping}
	\only<2>{\mappic{1} \quad One Layers}
	\only<3>{\mappic{2} \quad Two Layers}
	\only<4>{\mappic{3} \quad Three Layers}
	\only<5>{\mappic{4} \quad Four Layers}
	\only<6>{\mappic{5} \quad Five Layers}
	\only<7>{\mappic{10} \quad Ten Layers}
	\only<8>{\mappic{20} \quad Twenty Layers}
	\only<9>{\mappic{40} \quad Forty Layers}
\end{itemize}
}

\frame[plain]{
\frametitle{Deep Gaussian Processes}
\begin{itemize}
	\item Only one degree of freedom in $x$ is being captured!
	\item Again following Radford's thesis, connect every layer to input:
	 \def\nodeseptwo{4cm}
\def\nodesize{.45cm}
\def\numhiddentwo{4}


%\newlength{\arrowsize}  
%\pgfarrowsdeclare{biggertip}{biggertip}{  
%  \setlength{\arrowsize}{1pt}  
%  \addtolength{\arrowsize}{.5\pgflinewidth}  
%  \pgfarrowsrightextend{0}  
%  \pgfarrowsleftextend{-5\arrowsize}  
%}{  
%  \setlength{\arrowsize}{0.4pt}  
%  \addtolength{\arrowsize}{.5\pgflinewidth}  
%  \pgfpathmoveto{\pgfpoint{-5\arrowsize}{4\arrowsize}}  
%  \pgfpathlineto{\pgfpointorigin}  
%  \pgfpathlineto{\pgfpoint{-5\arrowsize}{-4\arrowsize}}  
%  \pgfusepathqstroke  
%} 

\begin{tabular}{ccc}
Standard deep net architecture & & Input-connected architecture \\
\bardist
\begin{tikzpicture}[draw=black]

    \tikzstyle{neuron}=[circle,minimum size=17pt, draw = black, fill = white, thick]
    \tikzstyle{input neuron}=[neuron, fill=green!50];
    \tikzstyle{output neuron}=[neuron, fill=red!50];
    \tikzstyle{hidden neuron}=[neuron, fill=blue!50];
    \tikzstyle{pile} =[ultra thick, ->, >=stealth', shorten <= 0.6cm, shorten >= 0.6cm, -biggertip, line width = 2pt];

    % Define the input layer node
    \coordinate (I) at (0, 0);


    % Define the hidden layer nodes
    \foreach \name / \y in {1,...,\numhiddentwo}
    {
        \coordinate (H-\name) at (\nodeseptwo*\y, 0);
    }

    % Connect every node            
    \foreach \name in {1,...,\numhiddentwo}
    {
	 \path[pile] (I) edge (H-\name) {};
         %\path[pile] (I) edge [bend left] (H-\name) {};
    }

    \draw (I) node[neuron, ultra thick] {};
    \draw (I) node[below = 0.5cm]  {$\vx$};

    % Draw the hidden layer nodes
    \foreach \name / \y in {1,...,\numhiddentwo}
    {
	\draw (H-\name) node[neuron, ultra thick]  {};
        \draw (H-\name) node[below = 0.34cm] {$\vf^{(\y)}$};
    }
\end{tikzpicture} & \hspace{2cm} &
\bardist
\begin{tikzpicture}[draw=black]
    \tikzstyle{neuron}=[circle,minimum size=17pt, draw = black, fill = white, thick]
    \tikzstyle{input neuron}=[neuron, fill=green!50];
    \tikzstyle{output neuron}=[neuron, fill=red!50];
    \tikzstyle{hidden neuron}=[neuron, fill=blue!50];
    \tikzstyle{pile} =[ultra thick, ->, >=stealth', shorten <= 0.6cm, shorten >= 0.6cm, -biggertip, line width = 2pt];

    % Define the input layer node
    \coordinate (I) at (0, 0);


    % Define the hidden layer nodes
    \foreach \name / \y in {1,...,\numhiddentwo}
    {
        \coordinate (H-\name) at (\nodeseptwo*\y, 0);
    }

    % Connect every node            
    \foreach \name in {1,...,\numhiddentwo}
    {
	 \path[pile] (I) edge (H-\name) {};
         \path[pile] (I) edge [bend left] (H-\name) {};
    }

    \draw (I) node[neuron, ultra thick] {};
    \draw (I) node[below = 0.5cm]  {$\vx$};

    % Draw the hidden layer nodes
    \foreach \name / \y in {1,...,\numhiddentwo}
    {
	\draw (H-\name) node[neuron, ultra thick]  {};
        \draw (H-\name) node[below = 0.34cm] {$\vf^{(\y)}$};
    }
\end{tikzpicture}
\end{tabular}

\end{itemize}
}


\newcommand{\onedsamplepiccon}[1]{
\includegraphics[width=0.7\columnwidth]{figures/1d_samples/latent_seed_0_1d_large_connected/layer-#1}
%\includegraphics[width=0.67\columnwidth]{figures/seed-0-map/latent_coord_map_layer_#1}%
}

\frame[plain]{
\frametitle{Deep Gaussian Processes}
\begin{itemize}
	\item Draws from input-connected one-dimensional deep GPs:
	\vspace{\baselineskip}
	\only<1>{\onedsamplepiccon{1}}
	\only<2>{\onedsamplepiccon{2}}
	\only<3>{\onedsamplepiccon{3}}
	\only<4>{\onedsamplepiccon{4}}
	\only<5>{\onedsamplepiccon{5}}
	\only<6>{\onedsamplepiccon{6}}
	\only<7>{\onedsamplepiccon{7}}
	\only<8>{\onedsamplepiccon{8}}
	\only<9>{\onedsamplepiccon{9}}
	\only<10>{\onedsamplepiccon{10}}
\end{itemize}
}


\newcommand{\gpdrawboxcon}[1]{
\setlength\fboxsep{0pt}
\fbox{
\includegraphics[width=0.67\columnwidth]{figures/deep_draws_connected/deep_sample_connected_layer#1}%
}}%

\frame[plain]{
\frametitle{Deep Gaussian Processes}
\begin{itemize}
	\item input-connected 2D to 2D warpings of a Gaussian density:
	\vspace{\baselineskip}
	\only<1>{\gpdrawbox{1}}%
	\only<2>{\gpdrawboxcon{2}}%
	\only<3>{\gpdrawboxcon{3}}%
	\only<4>{\gpdrawboxcon{4}}%
	\only<5>{\gpdrawboxcon{5}}%
	\only<6>{\gpdrawboxcon{6}}%
\end{itemize}
}




\frame[plain]{
\frametitle{Deep Gaussian Processes}
\begin{itemize}
	\item Showing the x that gave rise to a particular y
	\item (i.e. decision boundaries) \\
	\vspace{\baselineskip}
	\only<1>{\mappic{0} \quad No warping}
	\only<2>{\mappiccon{2} \quad Two Layers}
	\only<3>{\mappiccon{10} \quad Ten Layers}
	\only<4>{\mappiccon{20} \quad Twenty Layers}
	\only<5>{\mappiccon{40} \quad Forty Layers}
\end{itemize}
}



\frame[plain]{
\frametitle{Dropout}
\begin{itemize}
	\item Dropout is a method for regularizing neural networks.
	\item Recipe:
	\begin{enumerate}
		\item randomly set half of feature activations to zero
		\item Double the remaining features activations
		\item Average over all possible ways of dropping out
	\end{enumerate}
	\item Gives robustness, since neurons can't depend on one another.
	\item What do we get when we do dropout in GPs?
\end{itemize}
}




\newcommand{\numhiddentwo}[0]{5}

\frame[plain]{
\frametitle{Dropout on Feature Activations}

%\vspace{0.5cm}
\begin{tabular}{c|c}
\hspace{-1cm}
\begin{minipage}{0.5\textwidth}
\begin{tikzpicture}[shorten >=1pt,->,draw=black!50, node distance=\layersep]
    \tikzstyle{every pin edge}=[<-,shorten <=1pt]
    \tikzstyle{neuron}=[circle,fill=black!25,minimum size=17pt,inner sep=0pt]
    \tikzstyle{input neuron}=[neuron, fill=green!30];
    \tikzstyle{output neuron}=[neuron, fill=red!30];
    \tikzstyle{hidden neuron}=[neuron, fill=blue!30];
    \tikzstyle{annot} = [text width=4em, text centered]

    % Draw the input layer nodes
    \foreach \name / \y in {1,...,\numdims}
    % This is the same as writing \foreach \name / \y in {1/1,2/2,3/3,4/4}
        \node[input neuron, minimum size=\nodesize
        %, pin=left:Input \#\y
        ] (I-\name) at (0,-\nodesep*\y) {$x_\y$};

    % Draw the hidden layer nodes
    \foreach \name / \y in {1,...,\numhiddentwo}
        \path[yshift=1.5cm]
            node[hidden neuron, minimum size=\nodesize] (H-\name) at (\layersep,-\nodesep*\y) {$\phi_\y(\vx)$};

    % Draw the output layer node
    \foreach \name / \y in {1,...,\numouts}
    	\node[output neuron, minimum size=\nodesize
    	%,pin={[pin edge={->}]right:Output }
    	] (O-\name) at (2*\layersep,-\nodesep*2) {$f(x)$};

    % Connect every node in the input layer with every node in the
    % hidden layer.
    \foreach \source in {1,...,\numdims}
        \foreach \dest in {1,...,\numhiddentwo}
            \path (I-\source) edge (H-\dest);

    % Connect every node in the hidden layer with the output layer
%    \foreach \source in {1,...,\numhiddentwo}
 %       \foreach \dest in {1,...,\numouts}
    \only<1,2,3,5,7,9> {\path (H-1) [thick] edge (O-1);}
    \only<1,3,4,6,8,10> {\path (H-2) [thick] edge (O-1);}
    \only<1,3,2,3,6,7,10> {\path (H-3) [thick] edge (O-1);}
    \only<1,2,2,4,5,8,9> {\path (H-4) [thick] edge (O-1);}
    \only<1,2,2,3,6,8,10> {\path (H-5) [thick] edge (O-1);}

    \only<11>{\path (H-2) [black,thick] edge (O-1);}
    \only<11>{\path (H-3) [black,thick] edge (O-1);}
    \only<11>{\path (H-5) [black,thick] edge (O-1);}

    % Annotate the layers
%    \node[annot,above of=I-1, node distance=\upnodedist] {Inputs};
%    \node[annot,above of=H-1, node distance=\upnodedist] {Hidden};
%    \node[annot,above of=O-1, node distance=\upnodedist] {Output};
\end{tikzpicture} 
\end{minipage}
&
\begin{minipage}{0.53\textwidth}
\only<1>{Original formulation:}
\only<2-10>{Remove units with probability \nicefrac{1}{2}:}
\only<11>{Double output variance:}

$$f(\vx) = \frac{\only<1-10>{1} \only<11>{{\color{green}2}}}{K} \sum_{i=1}^K \only<2-11>{{\color{red}r_i}} \alpha_i \phi_i(\vx) \quad \only<2-11>{{\color{red}r_i} \simiid \textnormal{Ber}(\nicefrac{1}{2})}$$
with any weight distribution,
$$\expectargs{}{\only<11>{{\color{green}2}}\only<2-11>{{\color{red}r_i}} \alpha_i} = 0, \quad \varianceargs{}{\only<11>{{\color{green}2}} \only<2-11>{{\color{red}r_i}} \alpha_i} = 
\only<2-10>{{\color{red}\frac{1}{4}}}
\only<11>{{\color{green}\frac{4}{4}}}\sigma^2$$
by CLT, gives a GP as $K \to \infty$
$$\cov \left[ \! \begin{array}{c} f(\vx) \\ f(\vx') \end{array} \! \right] \to 
\only<2-10>{{\color{red}\frac{1}{4}}}
\only<11>{{\color{green}\frac{4}{4}}}
\frac{\sigma^2}{K}\sum_{i=1}^K \phi_i(\vx)\phi_i(\vx')$$
\end{minipage}
  \end{tabular}
}




\frame[plain]{
\frametitle{Dropout on Feature Activations}
\begin{itemize}
	\item Dropout on feature activations gives same GP
	\begin{itemize}
	\item Averaging the same model doesn't do anything
	 \end{itemize}
	\item GPs were doing dropout all along?	\smiley
	\item GPs strange because any one feature doesn't matter.
	\item Is there a better way to drop out features that would lead to robustness?
\end{itemize}
}


\frame[plain]{
\frametitle{Dropout on Inputs}
\begin{itemize}
%	\item Let $\vx$ be $D$ dimensional.
	\item Let $k(\vx, \vx') = \prod_{d=1}^D k_d(\vx_d, \vx_d')$
	\item Exact averaging over all dropouts is a mixture of GPs:$$ p(f(\vx)))= \frac{1}{2^D} \sum_{\vr \in \{0,1\}^D}  \gp \left(0, \prod_{d=1}^D k_d(\vx_d, \vx_d')^{r_d} \right)$$
	\item Probably a nice model, but presumably intractable.
	\item In neural net literature, exponentially-many dropout models are averaged over in tractable ways.
	\item For instance, dropout mixture has same covariance as $$ f(\vx) \sim \gp \left(0, \frac{1}{2^D} \sum_{\vr \in \{0,1\}^D}  \prod_{d=1}^D k_d(\vx_d, \vx_d')^{r_d} \right)$$
\end{itemize}
}


\frame[plain]{
\frametitle{Dropout on Inputs}
$$ f(\vx) \sim \gp \left(0, \sum_{\vr \in \{0,1\}^D}  \prod_{d=1}^D k_d(\vx_d, \vx_d')^{r_d} \right)$$
\begin{itemize}
	\item That's exactly additive GPs!  Kernel isocontours look like:
	\hspace*{-8pt}\makebox[\linewidth][c]{
\centering
\begin{tabular}{cccc}
\hspace{-0.2in} \includegraphics[width=0.27\textwidth]{figures/3d_add_kernel_1.pdf} &
\hspace{-0.2in} \includegraphics[width=0.27\textwidth]{figures/3d_add_kernel_2.pdf} &
\hspace{-0.2in} \includegraphics[width=0.27\textwidth]{figures/3d_add_kernel_3.pdf} & 
\hspace{-0.2in} \includegraphics[width=0.27\textwidth]{figures/3d_add_kernel_321.pdf}\\
%1st order interactions & 2nd order interactions & 3rd order interactions & All interactions \\
$k_1 + k_2 + k_3$ & $k_1k_2 + k_2k_3 + k_1k_3$ & $k_1k_2k_3$ & all terms\\
%& & (Squared-exp kernel) & (Additive kernel)\\
\end{tabular}
}
	\item Can compute all $2^D$ terms in $\mathcal{O}(D^2)$
	\item lots of functions, each only depends on some of the inputs
\end{itemize}
}



\setbeamertemplate{background canvas}{\begin{tikzpicture}\node[opacity=.1]{\includegraphics [width=\paperwidth]{figures/map_connected/latent_coord_map_layer_40}};\end{tikzpicture}}

\frame[plain]{
\frametitle{Summary}
\begin{itemize}
	\item At least two different ways to make GPs deep:
	\begin{itemize}
		\item composing kernels (inference easy, have to specify kernel)
		\item composing GPs (inference is hard, but can learn structure)
	\end{itemize}
%	\vspace{\baselineskip}
	\item Kernel learning is a form of representation learning
	\vspace{\baselineskip}
	\item Building priors over functions can shed light on architectures choices or initialization strategies in a data-independent way.
	\item What sorts of structures do we want to be able to learn? 
	\vspace{\baselineskip}
	\item We can bring tricks from the neural net literature back to GPs. (still need to try translation-invariant image kernels!)	
%	\item Open questions:
%	\begin{itemize}
%		\item When is discrete kernel search a good way to find representations?
%		\item What could we do if we could compute the marginal likelihood of a neural net?
%	\end{itemize}
\end{itemize}
	\pause
	\centering
	{
		\hfill
		{\color{blue} Thanks!}
				\hfill
	}
}


\end{document}

